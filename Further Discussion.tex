We can extend this thesis further. We can further check(study) like in Csikos's paper [7], that the differentiability condition can be replaced by continuity.\\


So now we came to the conclusion of this thesis. We proceeded this thesis by dividing this into two main parts i.e. (1) Literature review and (2) Main results. We started with the introduction of Kneser and Poulsen conjecture and talked about its importance. We described the configurationss of points under which the conjecture is constructed. We then moved to the history part. We noticed lots of Mathematicians from long time ago has been interested in this problem. We see solving this problem has different implications and helps us to think strengthen our thinking about higher dimensional space. In our thesis, we mainly followed the two papers one by Csikos[Csi98] and another by Bezdek and Conelly[BCO2]. We already talked about the importance of those paper and what has been done there. Our main problem was to follow the same methods like Csikos did and find out the formula for intersection of the balls instead of union. 

Our main results comes from the structure of the Voronoi diagram. In intersection, we take farthest Voronoi point, so if we construct in figure, the outward normal vector field from the wall of separation is in inwards towards the Voronoi cell being considered , so scaled by negative value. This will gives us negative result. There are some calculations related to t-derivatives and computing integrals. We also understood the inequalities in the lemma and understood corollaries.  We found studying the history of the development of the problem very interesting.  At last, we calculated the formula for finding the volume of intersection of balls.