
Let us remember our configuration from first page of this thesis in Introduction section. Let us suppose,we have  $\mathbf{r} \in \mathbf{R}_{+}^{N}$ and  there is a smooth curve $\mathbf{P}:(a, b) \rightarrow \mathbf{R}^{n N}$ where the motion of N balls's centers  is defined by $\boldsymbol{P}(t)=\left(\boldsymbol{p}_{1}(t), \ldots, \boldsymbol{p}_{N}(t)\right) \in \mathbb{E}^{n}$.We note smooth curve is many times differentiable infinitely. This is the second part of our thesis. Our goal here is to find a proof for the formula for derivative of function   
$V(t)=V(\mathbf{P}(t), \mathbf{r})$. Below we will talk about some lemma and corollary that can be used as a tool for our main problem. 


\section{Lemma} We can define,   L-infinity norm as a vector norm for a vector $\boldsymbol{P}=(p_1,p_2, \dots p_n)$ to be  $\|\mathbf{P}\|=\max _{i}\left|\mathbf{p}_{i}\right|$ on $\mathbf{R}^{n N}$ . Then, if we have given any fixed $\mathbf{r} \in \mathbf{E}_{+}^{N}$, we have the following (for $\|\mathbf{P}-\mathbf{Q}\| \rightarrow 0$ )
 $$
|V(\mathbf{P}, \mathbf{r})-V(\mathbf{Q}, \mathbf{r})|=O(\|\mathbf{P}-\mathbf{Q}\|)
$$
Note: Here $\boldsymbol{Q}=\left(\boldsymbol{q}_{1}, \ldots, \boldsymbol{q}_{N}\right)$  is just another configuration like $\boldsymbol{P}$\\



 












Proof:\\ Let us d denote by $\varepsilon$ the distance $\|\mathbf{P}-\mathbf{Q}\|$ by $\varepsilon$, then we can get these below estimations(using bounds) for the difference in volume.
$$
\begin{aligned}
& |V(\mathbf{P}, \mathbf{r})-V(\mathbf{Q}, \mathbf{r})| \\
& \quad \leq \max \left\{V\left(\bigcup_{i=1}^{N} B\left(\mathbf{p}_{i}, r_{i}+\varepsilon\right) \backslash B\left(\mathbf{p}_{i}, r_{i}\right)\right), V\left(\bigcup_{i=1}^{N} B\left(\mathbf{q}_{i}, r_{i}+\varepsilon\right) \backslash B\left(\mathbf{q}_{i}, r_{i}\right)\right)\right\} \\
\end{aligned}
$$
In this case, the difference between the volume of two sets $B\left(\mathbf{p}_{i}, r_{i}+\varepsilon\right) \backslash$ and $ B\left(\mathbf{p}_{i}, r_{i}\right)$ and $B\left(\mathbf{p}_{i}, r_{i}+\varepsilon\right) \backslash$ and $ B\left(\mathbf{p}_{i}, r_{i}\right)$ bounds the difference between difference between $V(\mathbf{P}, \mathbf{r})$ and $V(\mathbf{Q}, \mathbf{r})$. Since we are taking the maximum value of that difference for all $i$, so this maximum value of difference act as upper bound for $|V(\mathbf{P}, \mathbf{r})-V(\mathbf{Q}, \mathbf{r})|$. This is how we get the first inequality in the lemma.\\
$$\quad \leq V(B(\mathbf{0}, 1)) \sum_{i=1}^{N}\left(\left(r_{i}+\varepsilon\right)^{n}-r_{i}^{n}\right)=O(\varepsilon)$$ 

The difference between the volume of two sets $B\left(\mathbf{p}_{i}, r_{i}+\varepsilon\right) \backslash$ and $ B\left(\mathbf{p}_{i}, r_{i}\right)$ as well as  $B\left(\mathbf{p}_{i}, r_{i}+\varepsilon\right) \backslash$ and $ B\left(\mathbf{p}_{i}, r_{i}\right)$. These both difference is bounded by the maximum possible volume of a set $\boldsymbol{V}(\boldsymbol{B}(0,1))$ multiplied by nth power of radii $\left(\left(r_{i}+\varepsilon\right)^{n}-r_{i}^{n}\right)$ .  This comes from the fact that is volume of n-dimensional ball is proportional to its (nth power)radius. \bold{$V_n \propto r^n$}. This is how we get the second inequality.

$\implies |V(\mathbf{P}, \mathbf{r})-V(\mathbf{Q}, \mathbf{r})|=O(\|\mathbf{P}-\mathbf{Q}\| $ which means that the difference in the volume is bounded by a constant time distance between their centers($\epsilon$)\\



\section{Corollary}
\bold{NOTATION: For our problem, we will consider from this point to end} , $\boldsymbol{Q}$ \bold{as the Taylor expansion's linear part.  }\\


Let us consider there is a infinitely many times differentiable curve(smooth) $\mathbf{P}:(a, b) \rightarrow \mathbf{E}^{n N}$ such that for  $t_{0} \in(a, b)$, we have  $\mathbf{Q}(t)=\mathbf{P}\left(t_{0}\right)+\left(t-t_{0}\right) \mathbf{P}^{\prime}\left(t_{0}\right)$ as the linear part of the Taylor expansion of $\mathbf{P}$ at $t_{0}$. Then,we can say, the function $V(t)=V(\mathbf{P}(t), \mathbf{r})$ is differentiable at $t_{0}$  for any fixed $\mathbf{r} \in \mathbf{E}_{+}^{N}$ if and only if the function $t \mapsto V(\mathbf{Q}(t), \mathbf{r})$ is differentiable at $t_{0}$. Also, if these 2 functions are differentiable at $t_{0}$, then we can say that their derivatives must be equal at $t_{0}$.\\
Proof:\\
As we have assumed that $\mathbf{P}$ is a smooth curve, we can expand its Taylor series around $t_0$ and we can get the following after the expansion:\\
$\mathbf{P}(t)=\mathbf{P}\left(t_{0}\right)+\left(t-t_{0}\right) \mathbf{P}^{\prime}\left(t_{0}\right)+O\left(\left(t-t_{0}\right)^{2}\right)$
If we take only linear part of the Taylor expansion, we get,\\
$\mathbf{Q}(t)=\mathbf{P}\left(t_{0}\right)+\left(t-t_{0}\right) \mathbf{P}^{\prime}\left(t_{0}\right)$
Finding the difference between them we get,\\
 $\|\mathbf{P}(t)-\mathbf{Q}(t)\|=O\left(\left(t-t_{0}\right)^{2}\right)$
 This asymptotic notation$\|\mathbf{P}(t)-\mathbf{Q}(t)\|=O\left(\left(t-t_{0}\right)^{2}\right)$
implies that the distance between $\|\mathbf{P}(t)-\mathbf{Q}(t)\|$ is on the order of $(t-t_{0})^2$ i.e. as $t \rightarrow t_0$ distance between them cannot grow faster than $(t-t_{0})^2$\\
Now, we apply the following from Lemma 1,
$$
|V(\mathbf{P}(t), \mathbf{r})-V(\mathbf{Q}(t), \mathbf{r})|=O(\|\mathbf{P}-\mathbf{Q}\|)
&=O \left (O (\left(t-t_{0}\right)^{2} \right)
&=O\left(\left(t-t_{0}\right)^{2}\right)
$$
This proves corollary.
Difference in the volumes between $V(\mathbf{P}(t), \mathbf{r})$ and $V(\mathbf{Q}(t)$ is
$O\left(\left(t-t_{0}\right)^{2}\right)$ implies that this volume $V(\mathbf{P}(t), \mathbf{r})$ is differentiable at $t_0$ iff the volume due its linear part $V(\mathbf{Q}(t)$ is differentiable at $t_0$ and they have same derivatives at $t_0$.\\












