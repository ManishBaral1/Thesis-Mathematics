\documentclass[12pt]{report}
\usepackage[utf8]{inputenc}
\usepackage{graphicx}
\graphicspath{ {images/} }
\usepackage{caption}
\usepackage{subcaption}
\usepackage{float}
\usepackage[width=150mm,top=35mm,bottom=25mm,bindingoffset=6mm]{geometry}
\usepackage{amsmath}


\usepackage{fancyhdr}
\pagestyle{fancyplain}% <- use fancyplain instead fancy
\fancyhf{}
\fancyhead[R]{\thepage}
\renewcommand{\headrulewidth}{0pt}
\setlength{\headheight}{14pt}


\usepackage[style=authoryear,sorting=none]{biblatex}
\addbibresource{references.bib}



\title{On the Volume of the intersection of balls}
\author{Manish Baral}
\date{31 January 2023}
\begin{document}
\pagenumbering{roman} 
\begin{titlepage}
    \begin{center}
        \vspace*{1cm}
        \Huge
        \textbf{On the Volume of the Intersection of Balls }
        \vspace{0.5cm}

       \Large
        by\\    
        \text{Manish Baral} \\
        \text{Thesis Advisor: Prof. Dr. Igors Gorbovickis}
        
        \vfill
        
        A Thesis submitted in partial fulfillment\\
        of the requirements for the degree of\\
        Bachelor's of Science in\\
        Mathematics\\
        \vspace{1.8cm}
        \Huge
        at the\\ \textbf{Constructor University Bremen}\\
        \\
        \vspace{1.0cm}
        \begin{flushleft}
         
       
        \end{flushleft}
           \end{center}
    \end{titlepage}
\fancyhf{} % clear all header and footer fields
\fancyhead[RO,R]{\thepage} %RO=right odd, RE=right even
\renewcommand{\headrulewidth}{0pt}

    \textbf{Abstract}
    
Our project title is "On the Volume of the intersection of balls " and our aim is to give Csikos formula for the intersection of balls. We mostly follow paper Csi98[7] by Csikos and BCO2[3] by Bezdek and Connelly.
The motivation of our problem is "Kneser and Poulsen Conjecture" which states that, in n-dimensional Euclidean space  $ \mathbb{E}^n$ , if we rearrange the   balls in such a way that distance between each pair of centers doesnot decrease, then volume of union of balls cannot decrease and volume of intersection of balls cannot increase. It was first proposed by Martin Kneser in 1955 and later in 1968 by Arne Poulsen. It has been proved for n=2 completely by Bezdek and Connelly in [3] but remains open for $n \geq 3$. It been proven under additional condition on the number of points N and the map $p_i \rightarrow q_i$  for arbitrary $\mathbb{E}^n$.In the paper by Bezdek and Connelly the main idea used is the Csikos formula which allows to calculate the derivative of volume of (union\intersection) of balls as linear combination of derivatives of the distances between their centers with (non-negative\negative) coefficients. The expression (formula) for the union of balls was already proved by Csikos in 1998[7]. The formula for intersection of balls can be obtained in similiar way as union with slight(sign) change. This proving Csikos formula for intersection case is the main problem we will be proving in this project.





\tableofcontents

\chapter{Introduction}
\pagenumbering{arabic} 

\section{Configurations}
Let us use the notation $\mathbf{E}^{n}$ for $n$-dimensional Euclidean space. Let us define the inner product by $\langle\mathbf{p} , \mathbf{q}\rangle =\sum_{i=1}^{n} p_{i}. q_{i}|$, norm by   $|\mathbf{p}|=\sqrt{\langle\mathbf{p}, \mathbf{p}\rangle}$, and metric(distance) by $d(\mathbf{p_i}, \mathbf{p_j})=|\mathbf{p_i}-\mathbf{p_j}|$.

For every points in the Euclidean space $\mathbf{p} \in \mathbf{E}^{n}$ , we give radius $r \in \mathbf{E}_{+}$, where set of positive real numbers is denoted by $\mathbf{E}_{+}$ which is just another notation for($\mathbb{R}^n$), Let us suppose  closed ball of radius $r$ and center $\mathbf{p}$ be denoted by $B(\mathbf{p}, r)$ , let us suppose boundary of the sphere is denoted by $S(\mathbf{p}, r)$
Now, let us to go the configuration of a system of $N$ balls in $\mathbf{E}^{n}$. Let us denote the system of centers by $\mathbf{P}=\left(\mathbf{p}_{\mathbf{1}}, \ldots, \mathbf{p}_{N}\right) \in \mathbf{E}^{n N}$ and let us denote the corresponding radii by $\mathbf{r}=\left(r_{1}, \ldots, r_{N}\right) \in \mathbf{E}_{+}^{N}$. If we are given the system of $\mathbf{P}$ and $\mathbf{r}$,    the union of the balls $B\left(\mathbf{p}_{\mathbf{i}}, r_{i}\right)$ is denoted by $B(\mathbf{P}, \mathbf{r})$ for $1 \leq i \leq N$ (i.e. $B(\mathbf{P}, \mathbf{r}) =\bigcup_{i=1}^{n} B\left(\mathbf{p}_{\mathbf{i}}, r_{i}\right)$) In the similiar way, the boundary of the domain $B(\mathbf{P}, \mathbf{r})$ is denoted by $S(\mathbf{P}), $ and the $n$-dimensional volume of the ball $B(\mathbf{P}, \mathbf{r})$ is denoted by $V(\mathbf{P}, \mathbf{r})$. 




Let us suppose there are two configurations of N points in  $\mathbb{E}^n$ $ \boldsymbol{P}=\left(\boldsymbol{p}_{1}, \ldots, \boldsymbol{p}_{N}\right) \in \mathbb{E}^{n}$ and $\boldsymbol{Q}=\left(\boldsymbol{q}_{1}, \ldots, \boldsymbol{q}_{N}\right) \in \mathbb{E}^{n}$, $\forall 1\leq i < j \leq N,  \left|\boldsymbol{p}_{i}-\boldsymbol{p}_{j}\right|\ \leq \left|\boldsymbol{q}_{i}-\boldsymbol{q}_{j}\right|$, then Q is expansion of P. \\
Let us suppose, there is a motion $\boldsymbol{P}(t)=\left(\boldsymbol{p}_{1}(t), \ldots, \boldsymbol{p}_{N}(t)\right) \in \mathbb{E}^{n}$ for all $1 \leq i \leq N$ and $0 \leq t \leq 1$ such that $\boldsymbol{P}(0)=\boldsymbol{P}$ and $\boldsymbol{P}(1)=\boldsymbol{Q}$. We call this motion continuous expansion if for $1 \leq i<j \leq N$ $\left|\boldsymbol{p}_{i}(t)-\boldsymbol{p}_{j}(t)\right|$ is monotonically increasing.\\


\section{Kneser Poulsen Conjectures}
\textbf{Kneser Poulsen Conjectures(KP)}( conjectured independently by Kneser[15] in 1955 and Poulsen[16] in  1954) for $r_{1}=\cdots=r_{N}$ ) states that if $\boldsymbol{Q}=\left(\boldsymbol{q}_{1}, \ldots, \boldsymbol{q}_{N}\right)$ is an arbitrary expansion of $\boldsymbol{P}=\left(\boldsymbol{p}_{1}, \ldots, \boldsymbol{p}_{N}\right)$ in $\mathbb{E}^{n}$( i.e. $\forall 1\leq i < j \leq N,  \left|\boldsymbol{p}_{i}-\boldsymbol{p}_{j}\right|\ \leq \left|\boldsymbol{q}_{i}-\boldsymbol{q}_{j}\right|$), then we get the following holds true\\ 
\begin{itemize}
\item Conjecture 1. 
$$
\operatorname{V}\left[\bigcup_{i=1}^{N} B\left(\boldsymbol{p}_{i}, r_{i}\right)\right] \leq {V}\left[\bigcup_{i=1}^{N} B\left(\boldsymbol{q}_{i}, r_{i}\right)\right]
$$
\item Conjecture 2. 
$$
\operatorname{V}\left[\bigcap_{i=1}^{N} B\left(\boldsymbol{p}_{i}, r_{i}\right)\right] \geq \operatorname{V}\left[\bigcap_{i=1}^{N} B\left(\boldsymbol{q}_{i}, r_{i}\right)\right] .
$$
\item still open for $n \geq 3$
\end{itemize}


 \section{Importance of KP}
  It is one of the main problem in topology(combinatorial) and in geometry.A new insight high-dimensional spaces's structure can be derived if one could prove this conjecture.This proof has lots of important implications  to the fields algebraic topology, geometry, group theory, and other different fields of mathematics, computer science and physics,etc. The problem itself is motivation for the  invention of new mathematical tools,ideas and techniques.
In the past attempts, many Mathematicians tried to solve this conjectures came up with the new mathematical ideas(techniques) like  introduction of new cohomology theories and the study of topological  complexity and Borsuk-Ulam Theorem. In our paper, we are also following a Csikos technique(formula) to calculate the volume. His formula was also motivated from Kneser and Poulsen conjecture.\\

NOTATIONS ABOUT $\mathbf{Q}$:
We have used  $\mathbf{Q} =(\mathbf{q}_1,\dots,\mathbf{q}_N)$ and $\mathbf{P}(1)=\mathbf{Q}$ as this configuration in first part of this thesis Intro, literature review,etc. In Lemma 3.1,  $\mathbf{Q}$ is just this configurations of points $\mathbf{Q} =(\mathbf{q}_1,\dots,\mathbf{q}_N)$. From Chapter 3, Corollary 3.2 and below, we have used the symbol  $\mathbf{Q}$( along with $t_0$) to denote the Taylor expansion's linear part/








































\chapter{Literature review and Main results}
\input{chapters/Literature review and Main results.tex}

\chapter{Formula for finding the derivative A}

Let us remember our configuration from first page of this thesis in Introduction section. Let us suppose,we have  $\mathbf{r} \in \mathbf{R}_{+}^{N}$ and  there is a smooth curve $\mathbf{P}:(a, b) \rightarrow \mathbf{R}^{n N}$ where the motion of N balls's centers  is defined by $\boldsymbol{P}(t)=\left(\boldsymbol{p}_{1}(t), \ldots, \boldsymbol{p}_{N}(t)\right) \in \mathbb{E}^{n}$.We note smooth curve is many times differentiable infinitely. This is the second part of our thesis. Our goal here is to find a proof for the formula for derivative of function   
$V(t)=V(\mathbf{P}(t), \mathbf{r})$. Below we will talk about some lemma and corollary that can be used as a tool for our main problem. 


\section{Lemma} We can define,   L-infinity norm as a vector norm for a vector $\boldsymbol{P}=(p_1,p_2, \dots p_n)$ to be  $\|\mathbf{P}\|=\max _{i}\left|\mathbf{p}_{i}\right|$ on $\mathbf{R}^{n N}$ . Then, if we have given any fixed $\mathbf{r} \in \mathbf{E}_{+}^{N}$, we have the following (for $\|\mathbf{P}-\mathbf{Q}\| \rightarrow 0$ )
 $$
|V(\mathbf{P}, \mathbf{r})-V(\mathbf{Q}, \mathbf{r})|=O(\|\mathbf{P}-\mathbf{Q}\|)
$$
Note: Here $\boldsymbol{Q}=\left(\boldsymbol{q}_{1}, \ldots, \boldsymbol{q}_{N}\right)$  is just another configuration like $\boldsymbol{P}$\\



 












Proof:\\ Let us d denote by $\varepsilon$ the distance $\|\mathbf{P}-\mathbf{Q}\|$ by $\varepsilon$, then we can get these below estimations(using bounds) for the difference in volume.
$$
\begin{aligned}
& |V(\mathbf{P}, \mathbf{r})-V(\mathbf{Q}, \mathbf{r})| \\
& \quad \leq \max \left\{V\left(\bigcup_{i=1}^{N} B\left(\mathbf{p}_{i}, r_{i}+\varepsilon\right) \backslash B\left(\mathbf{p}_{i}, r_{i}\right)\right), V\left(\bigcup_{i=1}^{N} B\left(\mathbf{q}_{i}, r_{i}+\varepsilon\right) \backslash B\left(\mathbf{q}_{i}, r_{i}\right)\right)\right\} \\
\end{aligned}
$$
In this case, the difference between the volume of two sets $B\left(\mathbf{p}_{i}, r_{i}+\varepsilon\right) \backslash$ and $ B\left(\mathbf{p}_{i}, r_{i}\right)$ and $B\left(\mathbf{p}_{i}, r_{i}+\varepsilon\right) \backslash$ and $ B\left(\mathbf{p}_{i}, r_{i}\right)$ bounds the difference between difference between $V(\mathbf{P}, \mathbf{r})$ and $V(\mathbf{Q}, \mathbf{r})$. Since we are taking the maximum value of that difference for all $i$, so this maximum value of difference act as upper bound for $|V(\mathbf{P}, \mathbf{r})-V(\mathbf{Q}, \mathbf{r})|$. This is how we get the first inequality in the lemma.\\
$$\quad \leq V(B(\mathbf{0}, 1)) \sum_{i=1}^{N}\left(\left(r_{i}+\varepsilon\right)^{n}-r_{i}^{n}\right)=O(\varepsilon)$$ 

The difference between the volume of two sets $B\left(\mathbf{p}_{i}, r_{i}+\varepsilon\right) \backslash$ and $ B\left(\mathbf{p}_{i}, r_{i}\right)$ as well as  $B\left(\mathbf{p}_{i}, r_{i}+\varepsilon\right) \backslash$ and $ B\left(\mathbf{p}_{i}, r_{i}\right)$. These both difference is bounded by the maximum possible volume of a set $\boldsymbol{V}(\boldsymbol{B}(0,1))$ multiplied by nth power of radii $\left(\left(r_{i}+\varepsilon\right)^{n}-r_{i}^{n}\right)$ .  This comes from the fact that is volume of n-dimensional ball is proportional to its (nth power)radius. \bold{$V_n \propto r^n$}. This is how we get the second inequality.

$\implies |V(\mathbf{P}, \mathbf{r})-V(\mathbf{Q}, \mathbf{r})|=O(\|\mathbf{P}-\mathbf{Q}\| $ which means that the difference in the volume is bounded by a constant time distance between their centers($\epsilon$)\\



\section{Corollary}
\bold{NOTATION: For our problem, we will consider from this point to end} , $\boldsymbol{Q}$ \bold{as the Taylor expansion's linear part.  }\\


Let us consider there is a infinitely many times differentiable curve(smooth) $\mathbf{P}:(a, b) \rightarrow \mathbf{E}^{n N}$ such that for  $t_{0} \in(a, b)$, we have  $\mathbf{Q}(t)=\mathbf{P}\left(t_{0}\right)+\left(t-t_{0}\right) \mathbf{P}^{\prime}\left(t_{0}\right)$ as the linear part of the Taylor expansion of $\mathbf{P}$ at $t_{0}$. Then,we can say, the function $V(t)=V(\mathbf{P}(t), \mathbf{r})$ is differentiable at $t_{0}$  for any fixed $\mathbf{r} \in \mathbf{E}_{+}^{N}$ if and only if the function $t \mapsto V(\mathbf{Q}(t), \mathbf{r})$ is differentiable at $t_{0}$. Also, if these 2 functions are differentiable at $t_{0}$, then we can say that their derivatives must be equal at $t_{0}$.\\
Proof:\\
As we have assumed that $\mathbf{P}$ is a smooth curve, we can expand its Taylor series around $t_0$ and we can get the following after the expansion:\\
$\mathbf{P}(t)=\mathbf{P}\left(t_{0}\right)+\left(t-t_{0}\right) \mathbf{P}^{\prime}\left(t_{0}\right)+O\left(\left(t-t_{0}\right)^{2}\right)$
If we take only linear part of the Taylor expansion, we get,\\
$\mathbf{Q}(t)=\mathbf{P}\left(t_{0}\right)+\left(t-t_{0}\right) \mathbf{P}^{\prime}\left(t_{0}\right)$
Finding the difference between them we get,\\
 $\|\mathbf{P}(t)-\mathbf{Q}(t)\|=O\left(\left(t-t_{0}\right)^{2}\right)$
 This asymptotic notation$\|\mathbf{P}(t)-\mathbf{Q}(t)\|=O\left(\left(t-t_{0}\right)^{2}\right)$
implies that the distance between $\|\mathbf{P}(t)-\mathbf{Q}(t)\|$ is on the order of $(t-t_{0})^2$ i.e. as $t \rightarrow t_0$ distance between them cannot grow faster than $(t-t_{0})^2$\\
Now, we apply the following from Lemma 1,
$$
|V(\mathbf{P}(t), \mathbf{r})-V(\mathbf{Q}(t), \mathbf{r})|=O(\|\mathbf{P}-\mathbf{Q}\|)
&=O \left (O (\left(t-t_{0}\right)^{2} \right)
&=O\left(\left(t-t_{0}\right)^{2}\right)
$$
This proves corollary.
Difference in the volumes between $V(\mathbf{P}(t), \mathbf{r})$ and $V(\mathbf{Q}(t)$ is
$O\left(\left(t-t_{0}\right)^{2}\right)$ implies that this volume $V(\mathbf{P}(t), \mathbf{r})$ is differentiable at $t_0$ iff the volume due its linear part $V(\mathbf{Q}(t)$ is differentiable at $t_0$ and they have same derivatives at $t_0$.\\















\chapter{Formula for finding the derivative B}



\section{Theorem} 
Let us suppose there is a infinitely many times differentiable (smooth) curve  $\mathbf{P}:(a, b) \rightarrow \mathbf{E}^{n N}$. Let $n \geq 2$ , $\mathbf{r} \in \mathbf{E}_{+}^{N}$. Let us assume $t_{0} \in(a, b)$ is such a way that  balls $B\left(\mathbf{p}_{i}\left(t_{0}\right), r_{i}\right)$ are distinct from each other.Then the function $V(t)=V(\mathbf{P}(t), \mathbf{r})$ is differentiable at $t_{0}$ and its derivative is given by the following formula.
$$
V^{\prime}\left(t_{0}\right)=\sum_{i=1}^{N} \int_{F_{i}}\left\langle\mathbf{n}_{i}, \mathbf{p}_{i}^{\prime}\left(t_{0}\right)\right\rangle d \mu_{i}
$$\\
where $F_{i}$ =$S\left(\mathbf{p}_{i}\left(t_{0}\right), r_{i}\right) \cap S\left(\mathbf{P}\left(t_{0}\right), \mathbf{r}\right)$\\
& $\mathbf{n}_{i}$ = the outer unit normal vector field on the sphere $S\left(\mathbf{p}_{i}\left(t_{0}\right), r_{i}\right)$\\
&$\mu_{i}$ = $(n-1)$-dimensional volume measure on $S\left(\mathbf{p}_{i}\left(t_{0}\right), r_{i}\right)$ \\

Proof. As we know that from Corollary 3.2 above, since $V(\mathbf{P}(t), \mathbf{r})$ and $V(\mathbf{Q}(t)$ have equal derivatives, we can replace the original $\mathbf{P}$ with its linear approximation( due to Taylor expansion) $\mathbf{Q}(t)=$ $\mathbf{P}\left(t_{0}\right)+\left(t-t_{0}\right) \mathbf{P}^{\prime}\left(t_{0}\right)$\\
We now want to embedd the n-dimensional Euclidean space  $\mathbf{E}^{n}$ with one extra time dimension. We can consider  $\mathbf{E}^{n+1}$ as the product of $\mathbf{E}^{n} \times \mathbf{E}$. Then ,a typical element of  this $(n+1)$-dimensional Euclidean space  can be written in this  form $(\mathbf{p}, t)$, where $\mathbf{p} \in \mathbf{E}^{n}, t \in \mathbf{E}$. Let us fix,  $t \in(a, b)$. Now, 
we  construct the  $\mathbf{E}^{n+1}$ dimensional version of $B\left(\mathbf{p}_{i}\left(t_{0}\right), r_{i}\right),S\left(\mathbf{p}_{i}\left(t_{0}\right), r_{i}\right) , S\left(\mathbf{P}\left(t_{0}\right), \mathbf{r}\right)$ and $F_{i}$\\
After that we  apply the very important Gauss-Ostrogradskii  divergence theorem for the union of the solid cylinders to calculate the surface integrals.
$$
\mathbf{B}_{i}^*(t)=\left\{(\mathbf{p}, \tau) \in \mathbf{E}^{n+1} \mid a \leq \tau \leq t, \mathbf{p} \in B\left(\mathbf{p}_{i}(\tau), r_{i}\right)\right\}
$$
NOTATION: Here notations can be confusing. For  $\mathbf{E}^{n}$ space, we have used the notations $B\left(\mathbf{p}_{i}\left(t_{0}\right), r_{i}\right),S\left(\mathbf{p}_{i}\left(t_{0}\right), r_{i}\right) , S\left(\mathbf{P}\left(t_{0}\right), \mathbf{r}\right)$ and $F_{i}$\\
And for  $\mathbf{E}^{n+1}$ space, we have used the similiar notation but little bit bold and with $*$ as superscript. They look   $\mathbf{B}^*\left(\mathbf{p}_{i}\left(t_{0}\right), r_{i}\right),\mathbf{S}^*\left(\mathbf{p}_{i}\left(t_{0}\right), r_{i}\right) , \mathbf{S}^*\left(\mathbf{P}\left(t_{0}\right), \mathbf{r}\right)$ and $\mathbf{F}_{i}^{*}$
We have a constant vector field(which has divergence zero or divergence free)
$$
\xi \equiv(\mathbf{0}, 1) = 0 \vec{i} + 1 \vec{j} 
$$
Divergence = $$\frac{d(0 \vec{i})}{d \Vec{i}} + \frac{d(1 \vec{j})}{d \Vec{j}}=0+0=0 $$\\








We know that intuitively Gauss-Ostrogradskii theorem(in vector calculus) says that the volume integral of divergence over region inside a surface and surface integral of vector field over a surface(closed) are equal. Let us consider $\Omega$ is a compact(i.e. bounded and closed) domain in $\mathbf{E}^{m}$ with piecewise(i.e. smoothness holds even if the domain is divided into subdomains) smooth boundary  is Let a smooth vector field(constant) on a neighborhood of $\Omega$ be donoted by $\partial \Omega, \xi$, then, we have the following from Gauss divergence theorem:
$$
\int_{\Omega} \operatorname{div} \xi d \lambda^{m}=\int_{\partial \Omega}\langle\mathbf{N}^*, \xi\rangle d \mu,
$$
where $\mathbf{N}^*$ = outer unit normal vector field on $\partial \Omega$ (a. e.  on smooth components of $\partial \Omega$ )\\
$\operatorname{div} \xi$ = divergence of $\xi$\\
$\lambda^{m}$ = $m$-dimensional Lebesgue measure\\
$\mu$ = $(m-1)$-dimensional volume measure on $\partial \Omega$. \\


\textbf{Further Describing our cylindrical domain}\\

We have our domain $\Omega(t)=\bigcup_{i=1}^{N} \mathbf{B}^*_{i}(t)$\\
The boundary of the domain $\partial \Omega(t)$  is  smooth  piecewise. (It is due to the fact that we changed  the motion $\mathbf{P}(t)$ to its linear motion(due to Taylor expansion) $\mathbf{Q}(t)$ which was possible due to corollary(3.2).  Life-tubes of 2 smoothly moving spheres may intersect with one another in very complicated way, generally.) Since, the boundary is piecewise smooth, we can use Gauss divergence theorem here. 

\textbf{Top and bottom}
The boundary $\partial \Omega(t)$ has 2 flat components:  bottom  and  top.  The outer unit normal vector field on the bottom $\mathbf{B}^*(\mathbf{Q}(a), \mathbf{r}) \times\{a\}$ is $-\xi$ and on the top $\mathbf{B}^*(\mathbf{Q}(t), \mathbf{r}) \times$ $\{t\}$ it is $\xi$.\\
\textbf{Lateral}
$$
\large\mathbf{S}^*(t)=\left\{(\mathbf{p}, \tau) \in \mathbf{E}^{n+1} \mid a \leq \tau \leq t, \mathbf{p} \in S(\mathbf{Q}(\tau), \mathbf{r})\right\}
$$
gives the lateral part of $\partial\Omega(t)$. Similiarly,we have,
$$
\mathbf{S}_{i}^*(t)=\left\{(\mathbf{p}, \tau) \in \mathbf{E}^{n+1} \mid a \leq \tau \leq t, \mathbf{p} \in S\left(\mathbf{q}_{i}(\tau), r_{i}\right)\right\}
$$
$\mathbf{F}_{i}^*(t)=\mathbf{S}^*(t) \cap \mathbf{S}_{i}^*(t)$

Also, we have ,  for $a \leq \tau \leq t$,  the set $F_{i}(\tau)=\mathbf{F}_{i}^*(t) \cap\left(\mathbb{E}^{n} \times\{\tau\}\right)$. 
Finally, we have the outer unit normal vector field of the hypersurface of cylinder $\mathbf{F}_{i}^*$ at $(\mathbf{p}, \tau) \in \mathbf{F}_{i}^*$ give by;
$$
\mathbf{N}^*= \frac{\left(\mathbf{n}_{i}(\mathbf{p}, \tau),-\left\langle\mathbf{n}_{i}(\mathbf{p}, \tau), \mathbf{p}_{i}^{\prime}\left(t_{0}\right)\right\rangle\right)}{\sqrt{1+\left\langle\mathbf{n}_{i}(\mathbf{p}, \tau), \mathbf{p}_{i}^{\prime}\left(t_{0}\right)\right\rangle^{2}}},
$$

Here we know that,  outer unit normal of the sphere $S\left(\mathbf{q}_{i}(\tau), r_{i}\right)$ at $\mathbf{p}$ is denoted by $\mathbf{n}_{i}(\mathbf{p}, \tau)$ 
















\textbf{Remark 1}: Where inner product come from?( in outer unit normal of hyperspace of cylinder)
$$
p(t)=(p_1(t),p_2(t),\dots p_n(t),\quad t  ) \in \mathbb{E}^{n+1}
$$
And its derivatives be,
$$
p^{'}(t)=(p_1^{'}(t),p_2^{'}(t),\dots p_n^{'}(t),\quad 1  ) 
$$
Let us denote the outer unit normal vector field of the hyperspace of cylinder by two components as follows:
$$
\mathbf{N}^* \in \mathbb{E}^{n+1} =(n_i\in \mathbb{E},n_t \in 1)
$$

$$\mathbf{N}^*.p^{'}(t) \quad \textbf{must be orthogonal}$$
$$
\mathbf{N}^*.p^{'}(t)=0
$$
$$
\text{or,} \quad \binom {n_i}{n_t}.\binom {x^{'}(t)}{1}=0
$$
$$
\text{or,} \quad n_t = -\big<   n_i, x^{'}(t)   \big>
$$
So, this is how the negative inner product gives one of the component of $\mathbf{N}^*$. We know,
$$
\mathbf{N}^* \in \mathbb{E}^{n+1} =(n_i,n_t )
$$
Normalizing $ n_i$ and $n_t$, we get(i.e. they are being divided by their magnitude)
$$
\mathbf{N}^*= \left( \frac{ \mathbf{n}_{i} } {\sqrt{1+\left\langle \mathbf{n}_{i}, \mathbf{p}_{i}^{\prime}\left(t_{0}\right)\right\rangle^{2}}},\frac{- \left\langle\mathbf{n}_{i}, \mathbf{p}_{i}^{\prime}\left(t_{0}\right)\right\rangle} {\sqrt{1+\left\langle \mathbf{n}_{i},\mathbf{p}_{i}^{\prime}\left(t_{0}\right)\right\rangle^{2}}} \right)
$$
$$
\mathbf{N}^*= \frac{\left(\mathbf{n}_{i}(\mathbf{p}, \tau),-\left\langle\mathbf{n}_{i}(\mathbf{p}, \tau), \mathbf{p}_{i}^{\prime}\left(t_{0}\right)\right\rangle\right)}{\sqrt{1+\left\langle\mathbf{n}_{i}(\mathbf{p}, \tau), \mathbf{p}_{i}^{\prime}\left(t_{0}\right)\right\rangle^{2}}},
$$\\
Note: We can further study about this normal using cylindrical co-ordinates and describing the tangent plane in more detail. We can also say that , if we consider two vectors in tangent plane, a vector perpendicular to both of them(given by cross product of those 2 vectors)  is normal $\mathbf{N}^*$ of that tangent plane.  























\textbf{Remark 2}
Note: This Remark 2 is relating to more advanced concepts of mathematics.
We noted here that tube $S\left(\mathbf{p}_{i}\left(t_{0}\right), r_{i}\right) \times[a, b]$ is diffeomorphic to $\mathbf{S}_{i}^*(t)$. We get the following, if we identify 2 spaces by diffeomorphism.

$$
(\mathbf{p}, \tau) \leftrightarrow\left(\mathbf{p}+\left(\tau-t_{0}\right) \mathbf{p}^{\prime}\left(t_{0}\right), \tau\right) \in \mathbb{S}_{i}(t),
$$
$\implies$ This is a linear map and linear maps are diffeomorphic ( i.e. if we identify those two space as manifolds*, the above given differentiable map is bijection and its inverse is differentiable as well) 



\textbf{Remark 3}
For all $\mathbf{p}$ belonging to $S\left(\mathbf{p}_{i}\left(t_{0}\right), r_{i}\right)$ and for all $\tau $ belonging to the interval $[a, b]$, we can express the $n$-dimensional volume measure  with $(n-1)$ dimensioanl volume measure $\mu_{i}$ and the one-dimensional Lebesgue measure $\lambda$ on $[a, b]$. The above $(n-1)$ dimensional volume measure $\mu_{i}$ is multiplied by a factor $\sqrt{1+\left\langle\mathbf{n}_{i}(\mathbf{p}, \tau), \mathbf{p}_{i}^{\prime}\left(t_{0}\right)\right\rangle^{2}}$ which depends upon derivative of $\mathbf{p}_i(t)$ at $t_0$ and the outer unit normal vector of sphere $n_i$
i.e.
$$
\sqrt{1+\left\langle\mathbf{n}_{i}(\mathbf{p}, \tau), \mathbf{p}_{i}^{\prime}\left(t_{0}\right)\right\rangle^{2}} \mu_{i} \times \lambda
$$



























We know that the vector field $\xi=(\mathbf{0}, 1)$ is divergence( =0) free, \textbf{the Gauss-Ostrogradskii divergence theorem gives the following result:} We start by calculating the surface integrals at top+bottom+lateral part of the cylinder. The surafce integral at the lateral part is given by integrating inner product of $\mathbf{N}^*$ and $\xi$ over the boundary. The calculation is straightforward. We computer inner product properly and differentiate on later part. In one step, we replace $ d \mu  $ by $\sqrt{1+\left\langle\mathbf{n}_{i}, \mathbf{p}_{i}^{\prime}\left(t_{0}\right)\right\rangle^{2} } d \mu_i * d \tau  $. It is due to Remark 3 (the idea of representing n-dimensional volume measure as product of 1-dimensional volume measure and (n-1) dimensional measure. Now, we proceed to application of Divergence theorem;
$$
\begin{aligned}
0 & =\int_{\Omega(t)} \operatorname{div} \xi d \lambda^{n+1}=\int_{\partial \Omega(t)}\langle\mathbf{N}^*, \xi\rangle d \mu \\
& =V(\mathbf{Q}(t), \mathbf{r})-V\left(\mathbf{Q}\left(t_{0}\right), \mathbf{r}\right)+\sum_{i=1}^{N} \int_{\mathbb{F}_{i}(t)}\left\langle \frac{\left(\mathbf{n}_{i}(\mathbf{p}, \tau),-\left\langle\mathbf{n}_{i}(\mathbf{p}, \tau), \mathbf{p}_{i}^{\prime}\left(t_{0}\right)\right\rangle\right)}{\sqrt{1+\left\langle\mathbf{n}_{i}(\mathbf{p}, \tau), \mathbf{p}_{i}^{\prime}\left(t_{0}\right)\right\rangle^{2}}}, (0,1) \right\rangle d \mu \\
& =V(\mathbf{Q}(t), \mathbf{r})-V\left(\mathbf{Q}\left(t_{0}\right), \mathbf{r}\right)+ \sum_{i=1}^N             \int_{\mathbb{F}_i(t)}\left\langle \left( \frac{ \mathbf{n}_{i} } {\sqrt{1+\left\langle \mathbf{n}_{i}, \mathbf{p}_{i}^{\prime}\left(t_{0}\right)\right\rangle^{2}}},\frac{- \left\langle\mathbf{n}_{i}, \mathbf{p}_{i}^{\prime}\left(t_{0}\right)\right\rangle} {\sqrt{1+\left\langle \mathbf{n}_{i},\mathbf{p}_{i}^{\prime}\left(t_{0}\right)\right\rangle^{2}}} \right),(0,1)\right\rangle d \mu \\
& =V(\mathbf{Q}(t), \mathbf{r})-V\left(\mathbf{Q}\left(t_{0}\right), \mathbf{r}\right)-\sum_{i=1}^{N} \int_{\mathbb{F}_{i}(t)} \frac{\left\langle\mathbf{n}_{i}, \mathbf{p}_{i}^{\prime}\left(t_{0}\right)\right\rangle}{\sqrt{1+\left\langle\mathbf{n}_{i}, \mathbf{p}_{i}^{\prime}\left(t_{0}\right)\right\rangle^{2}}} d \mu \\
& =V(\mathbf{Q}(t), \mathbf{r})-V\left(\mathbf{Q}\left(t_{0}\right), \mathbf{r}\right)-\sum_{i=1}^{N} \int_{\mathbb{F}_{i}(t)} \frac{\left\langle\mathbf{n}_{i}, \mathbf{p}_{i}^{\prime}\left(t_{0}\right)\right\rangle}{\sqrt{1+\left\langle\mathbf{n}_{i}, \mathbf{p}_{i}^{\prime}\left(t_{0}\right)\right\rangle^{2}}}  \sqrt{1+\left\langle\mathbf{n}_{i}, \mathbf{p}_{i}^{\prime}\left(t_{0}\right)\right\rangle^{2} } d \mu_i * d \tau \\
& \text{or,}\quad 0 =V(\mathbf{Q}(t), \mathbf{r})-V\left(\mathbf{Q}\left(t_{0}\right), \mathbf{r}\right)-\sum_{i=1}^{N} \int_{a}^{t}\left(\int_{F_{i}(\tau)}\left\langle\mathbf{n}_{\mathbf{i}}, \mathbf{p}_{i}^{\prime}\left(t_{0}\right)\right\rangle d \mu_{i}\right) d \tau
\end{aligned}
$$
Differentiating both sides with respect to $t$ at $t_0$, we get 
$$
\frac{d(V(\mathbf{Q}(t), \mathbf{r})-V\left(\mathbf{Q}\left(t_{0}\right), \mathbf{r}\right)}{d \tau} = \frac{d}{d \tau} \sum_{i=1}^{N} \int_{0}^{t} \left( \int_{F_{i}(\tau)} \left\langle \mathbf{n}_i,\mathbf{p}_{i}^{\prime}(t_0) \right\rangle  d \mu_i  \right) d \tau
$$
Note: $\tau = t- t_0$
$$\implies
V^{\prime}\left(t_{0}\right)=\sum_{i=1}^{N} \int_{F_{i}}\left\langle\mathbf{n}_{i}, \mathbf{p}_{i}^{\prime}\left(t_{0}\right)\right\rangle d \mu_{i}
$$

We know, if $n \geq 2$, 
The dependency is continuous in a neighborhood of $t_{0}$.
i.e. the integrals
$$
\int_{F_{i}(\tau)}\left\langle\mathbf{n}_{\mathbf{i}}, \mathbf{p}_{i}^{\prime}\left(t_{0}\right)\right\rangle d \mu_{i}
$$
depends on $\tau$ in the neighborhood of any $\tau_{0}$  continuously where the balls $B\left(\mathbf{p}_{i}\left(\tau_{0}\right), r_{i}\right)$ are different


























































\chapter{Voronoi Decomposition}
Voronoi(and Dirichlet)Decomposition  is a technique that divides a space into regions on the basis  of distances to a set of considered points. In chapter 4 above, we find out the method to find derivative of the function $V(t)=V(\mathbf{P}(t), \mathbf{r})$. There, we got a final formula. If we need to compute the volume of Union or intersection of balls , we first need to to divide the space using Dirichlet Voronoi decomposition and we apply Gauss-Ostrogradskii divergence theorem to those decomposed voronoi region(cells).
We will give introduction of Voronoi decomposition below.\\

\begin{itemize}


    \item Let us begin by defining the power of a point.The power of a point $\mathbf{a}$ with respect to a ball $B(\mathbf{p}, r)$ can be defined as this real number $|\mathbf{p}-\mathbf{a}|^{2}-r^{2}$. Then, when we are given  $N$ spheres by assigning the radii $\mathbf{r}$ to the centers $\mathbf{P}$. The, we can define $K_{i}(\mathbf{a})=\left|\mathbf{p}_{\mathbf{i}}-\mathbf{a}\right|^{2}-r_{i}^{2}$ as the ith  power of $\mathbf{a}$ wrt $i$ th ball. 


    
    \item  
    If $\mathbf{p}_{\mathbf{i}} \neq \mathbf{p}_{j}$,  $1 \leq i<j \leq N$, then  a half-space bounded by hyperplane is defined by this inequality $K_{i}(\mathbf{a}) \leq K_{j}(\mathbf{a})$\\ 
    The following  is the equation of the hyperplane. 
    $$
2\left\langle\left(\mathbf{p}_{\mathbf{i}}-\mathbf{p}_{j}\right), \mathbf{a}\right\rangle=\mathbf{p}_{i}^{2}-\mathbf{p}_{j}^{2}+r_{j}^{2}-r_{i}^{2} .
$$\\
Remark: Since, at hyperplane $K_{i}(\mathbf{a})=K_{j}(\mathbf{a})$,therefore this equation of the hyperplane is given by $\left|\mathbf{p}_{\mathbf{i}}-\mathbf{a}\right|^{2}-r_{i}^{2}=\left|\mathbf{p}_{\mathbf{j}}-\mathbf{a}\right|^{2}-r_{j}^{2}$


Case 1: For any pair $1 \leq i<j \leq N$, if $\mathbf{p}_{i}=\mathbf{p}_{j}$ (same center), then $K_{i}(\mathbf{a})-K_{j}(\mathbf{a})=r_{j}^{2}-r_{i}^{2}$ is constant.\\
Case 2: If balls are distinct,we determine Voronoi decomposition.


If the balls are distinct, then the balls determine a decomposition of the space into $N$ possibly unbounded or empty polyhedral domains. The $i$ th cell of the decomposition is given as the closure of the open domain















\item Then, let us suppose we are given configuration $ \mathbf{P}=(p_1, \dots ,p_N) \in  \mathbb{E}^n$ and radii is given by $r_1, \dots, r_N$. Then the following sets gives the  $i$th cell decomposition. This is called as Dirichlet Voronoi Decomposition:\\ 
$$
C_{i}=\left\{\mathbf{a} \in \mathbf{E}^{n} \mid K_{i}(\mathbf{a})\leq K_{j}(\mathbf{a}) \text { for all } 1 \leq j \leq N, j \neq i\right\} \text {. }
$$
$$
C_{i}=\left\{\mathbf{a} \in \mathbf{E}^{n} \left|\mathbf{p}_{\mathbf{i}}-\mathbf{a}\right|^{2}-r_{i}^{2}\leq \left|\mathbf{p}_{\mathbf{j}}-\mathbf{a}\right|^{2}-r_{j}^{2}  \text { for all } 1 \leq j \leq N, j \neq i\right\}
$$
We call the set $C_{i}$ as nearest point Dirichlet Voronoi region of points $\mathbf{a}$. This corresponds with Union  problem. A very good argument about how this fits the Union version of this problem is done by Edelsbrunner[17] in his paper.  




$$
C^{i}=\left\{\mathbf{a} \in \mathbf{E}^{n} \mid K_{i}(\mathbf{a}) \geq K_{j}(\mathbf{a}) \text { for all } 1 \leq j \leq N, j \neq i\right\} \text {. }
$$
$$
C^{i}=\left\{\mathbf{a} \in \mathbf{E}^{n} \left|\mathbf{p}_{\mathbf{i}}-\mathbf{a}\right|^{2}-r_{i}^{2} \geq \left|\mathbf{p}_{\mathbf{j}}-\mathbf{a}\right|^{2}-r_{j}^{2} \text { for all } 1 \leq j \leq N, j \neq i\right\}
$$
We call the set $C^{i} $ as farthest point Dirichlet Voronoi cells. This corresponds with the intersection version of our problem. More  reasearch about this region is done by Seidel[18]. This decomposition for the intersection is what we are using in our problem as we need to find the volume of intersection of the balls. 

\item When this Voronoi decomposition is induced in the ball $B(\mathbf{P}, \mathbf{r})$, we call it  (trauncated) nearest and farthest Voronoi cells for union and intersection respectively.

$$
\tilde{C}_{i}=\bar{C}_{i} \cap B(\mathbf{P}, \mathbf{r})=\bar{C}_{i} \cap B\left(\mathbf{p}_{i}, r_{i}\right),
$$
Here, $\bar{C}$ = closure of the set $C$.
$$
\tilde{C}^{i}=\bar{C}^{i} \cap B(\mathbf{P}, \mathbf{r})=\bar{C}_{i} \cap B\left(\mathbf{p}_{i}, r_{i}\right),
$$
Here, $\bar{C}$ = closure of the set $C$.




\item We mainly talk about the intersection part now because that is what we are doing later.  The term  $W^{i j}$ appear in our final formula. It is called the wall. We are considering the intersection case, so, the wall $W^{i j}$ between the cells $\tilde{C}^{i}$ and $\tilde{C}^{j}$  can be defined as the intersection of $ith$ and $jth $ vornoi regions i.e. $W^{i j}=\tilde{C}^{i} \cap \tilde{C}^{j}$. At $K_{i}(\mathbf{a})=K_{j}(\mathbf{a})$, the $W^{i j}$ is not empty. 
For easiness of notation and to show  relation of $\tilde{C}^{i}$ and $W^{i j}$ with $(\mathbf{P}, \mathbf{r})$, we normally use $\tilde{C}^{i}(\mathbf{P}, \mathbf{r})$ and $W^{i j}(\mathbf{P}, \mathbf{r})$


\item There are many easily verified properties of Voronoi diagrams. We will use mainly property(iv) given in section 3 of [BCO2] in our problem of intersection. The property says, when $W^{i j}(\mathbf{P},\mathbf{r})$ is non empty, the vector $\mathbf{p}_j - \mathbf{p}_i $ is a scalar multiple  (and negative in sign ) of the outward pointing normal to boundary of $C^i(\mathbf{r})$ at $W^{i j}(\mathbf{P}, \mathbf{r})$. If we draw  the outer unit normal vector of of region $C^{i}$ in all directions, one relating with  $C^{i}$ will be in opposite direction to the cell $C^{i}$.

\end{itemize}





\chapter{Formula for Intersection}
\section{Theorem} Let us suppose there is a smooth curve $\mathbf{P}:(a, b) \rightarrow \mathbf{E}^{n N}$ for $n \geq 2$ and $\mathbf{r} \in \mathbf{E}_{+}^{N}$. Now, let $t_{0} \in(a, b)$ is such that the centers $\mathbf{p}_{i}\left(t_{0}\right)$ are different from each other. Then the function $V(t)=V(\mathbf{P}(t), \mathbf{r})$ is differentiable at $t_{0}$ and its derivative must be equal to
$$ V^{'}(t_0)=\frac{d}{d t} V(t_0)=\sum_{1 \leq i<j \leq N} -d_{i j}^{\prime} \operatorname{V}_{n-1}\left[W^{i j}\left(\boldsymbol{p}_{i}(t_0), r_{i}\right)]\right $$
where, $V_{n-1}$ denotes the $(n-1)$-dimensional volume\\
$d_{i j}(t_0)=d\left(\mathbf{p}_{i}(t_0), \mathbf{p}_{j}(t_0)\right)=\left|\boldsymbol{p}_{i}(t)-\boldsymbol{p}_{j}(t_0)\right|$\\
$d_{i j}^{\prime}$ = $t$-derivative of $d_{i j}$\\
$W^{i j}=\tilde{C}^{i} \cap \tilde{C}^{j}$
Remark: We note that the union version of the same problem was proved by Csikos in (Theorem 4.1 of his paper [7] and it is also mentioned in the Main result(literature review) section of this thesis. We have used his proof as the model for our proof. With necessary adjustment we can reach the negative sign. The adjustment is that, from above section Voronoi diagram, we came to know that the outer unit normal of non empty wall in case of intersection is nagative. It is due to the structure or construction of the Voronoi region $C^{i}$ as in intersection we take farthest part.\\

Proof. From Theorem $4.1$ at chapter 4 of this thesis above  , we came to know that $V$ is differentiable at $t_{0}$ and $V^{\prime}\left(t_{0}\right)$ can be given by that formula from above. Originally, this formula was proved by Csikos again in (Theorem 2.3 of [7]). We continue from last part of Chapter 4 of this thesis. We compute the following integral.
$$
\int_{F_{i}}\left\langle\mathbf{n}_{i}, \mathbf{p}_{i}^{\prime}\left(t_{0}\right)\right\rangle d \mu_{i}
$$


As it is our case of intersection, we apply the Gauss-Ostrogradskii Divergence theorem  to the domain $\tilde{C}^{i}$ which is the voronoi region of intersection of $i$th cell. The constant vector field is given by $\xi_{i} \equiv \mathbf{p}_{i}^{\prime}\left(t_{0}\right)$

If we look at the structure, the cell $\tilde{C}^{i}$ is bounded by the spherical domain $F_{i}$ and the walls $W^{i j}$. The outer unit normal of a nonempty wall $W^{i j}$ is (The sign will be negative in case of intersection due to construction and fact that as voronoi region $\tilde{C}^{i}$ lies on farther side from ith ball.)
$$
\frac{-(\mathbf{p}_{j}\left(t_{0}\right)-\mathbf{p}_{i}\left(t_{0}\right))}{d_{i j}\left(t_{0}\right)} .
$$

We know that the constant vector field $\xi_{i}$ has divergence 0,so  we get that,
$$
0=\int_{\tilde{C}^{i}} \operatorname{div} \xi_{i} d \lambda^{n}=\int_{\partial \tilde{C}_{i}}\left\langle\mathbf{N}^*, \xi_{i}\right\rangle d \mu,
$$

here, $\mu$ =  $(n-1)$-dimensional volume measure on $\tilde{C}^{i}$.\\
     $\mathbf{N}^*$ = outer unit normal field
 We can calculate further and we get,
$$
\int_{\partial \tilde{C}^{i}}\left\langle\mathbf{N}^*, \xi_{i}\right\rangle d \mu=\int_{F_{i}}\left\langle\mathbf{n}_{i}, \mathbf{p}_{i}^{\prime}\left(t_{0}\right)\right\rangle d \mu_{i}+\sum_{\substack{j=1 \\ j \neq i}}^{N}\left\langle\frac{-(\mathbf{p}_{j}\left(t_{0}\right)-\mathbf{p}_{i}\left(t_{0}\right))}{d_{i j}\left(t_{0}\right)}, \mathbf{p}_{i}^{\prime}\left(t_{0}\right)\right\rangle V_{n-1}\left(W^{i j}\right)
$$
Remark:In the above step, we noticed that since surface integral at the Top and bottom part cancel each other since one in on the direction of constant vector field and another is on the opposite direction of constant vector field. For the lateral part we need to compute the surface integrals of spherical part and wall part, which is given below. \\
Remark: We note that for the spherical part surface integral, we can use the result from Chapter 4 , we have calculated above.Futhermore, we get the following result. 
$$
\int_{F_{i}}\left\langle\mathbf{n}_{i}, \mathbf{p}_{i}^{\prime}\left(t_{0}\right)\right\rangle d \mu_{i}=\sum_{\substack{j=1 \\ j \neq i}}^{N}\left\langle\frac{-(\mathbf{p}_{i}\left(t_{0}\right)-\mathbf{p}_{j}\left(t_{0}\right)}{d_{i j}\left(t_{0}\right))}, \mathbf{p}_{i}^{\prime}\left(t_{0}\right)\right\rangle V_{n-1}\left(W^{i j}\right) .
$$
If we sum the above  for all $i$, we obtain the following calculations.
$$
\begin{aligned}
 & =\sum_{i=1}^{N} \int_{F_{i}}\left\langle\mathbf{n}_{i}, \mathbf{p}_{i}^{\prime}\left(t_{0}\right)\right\rangle d \mu_{i} \\
& =\sum_{1 \leq i<j \leq N}\left\langle\frac{-(\mathbf{p}_{i}\left(t_{0}\right)-\mathbf{p}_{j}\left(t_{0}\right))}{d_{i j}\left(t_{0}\right)}, \mathbf{p}_{i}^{\prime}\left(t_{0}\right)-\mathbf{p}_{j}^{\prime}\left(t_{0}\right)\right\rangle V_{n-1}\left(W^{i j}\right) \\
& =\sum_{1 \leq i<j \leq N} -d_{i j}^{\prime}\left(t_{0}\right) V_{n-1}\left(W^{i j}\right)
\end{aligned}
$$
$$
\implies V^{\prime}\left(t_{0}\right) =\sum_{i=1}^{N} \int_{F_{i}}\left\langle\mathbf{n}_{i}, \mathbf{p}_{i}^{\prime}\left(t_{0}\right)\right\rangle d \mu_{i} =\sum_{1 \leq i<j \leq N} -d_{i j}^{\prime}\left(t_{0}\right) V_{n-1}\left(W^{i j}\right)
$$
Hence, this is the main result of our problem that was proved.





Note: Reaching the final t-derivative term might be tricky. We compute the inner product the following ways to reach the final result.\\
Firstly, we square the term $d_{i j} $ which lead us to taking inner product of ($p_i-p_j$) with itself.

    $$d_{i j}^2 = || p_{i}^{2}- p_{j}^{2} ||= \left\langle p_i - p_j,p_i-p_j\right\rangle $$
    Now we differentiate $d_{i j}^2 $ with respect to time(t) 
    $$ 
    \frac{d}{dt}(d_{i j}^2) = 2 d_{i j} d^{'}_{i j}
    $$
From above two relations we now know that derivative of
$$d_{i j}^2  = \frac{d}{dt} \left\langle p_i-p_j, p_i -p_j \right\rangle  $$ 
From above last two relations, we can write the following:
$$
2 d_{i j} d^{'}_{i j} = \frac{d}{dt} \left\langle p_i-p_j, p_i -p_j \right\rangle 
$$
$$
d^{'}_{i j}=\frac{\frac{d}{dt} \left\langle p_i-p_j, p_i -p_j \right\rangle 
}{2 d_{i j}}
$$

Let $(V,⟨⋅,⋅⟩)$
be  space of inner product which is of finite dimension and $g,h:\mathbb{R} \rightarrow{V}$ be differentiable functions
,then, we know that, from the rules of differentiation of inner product.
$$
\frac{d}{dt}\left\langle g,h \right\rangle = \left\langle g(t) , h^{'}(t) \right\rangle + \left\langle f^{'}(t), h(t) \right\rangle
$$
\text{Now, using this rules, we get,}
$$
d^{'}_{i j}= \frac{\left\langle p_i - p_j, (p_i- p_j)^{'} \right\rangle + \left\langle (p_i - p_j)^{'},p_i-p_j\right\rangle}{2 d_{i j}}
$$
\text{Again, using the Symmetric property of inner product:} $$\left\langle u,v \right\rangle=\left\langle v,u \right\rangle$$ for  vectors $u$, $v$. 
$$
or, d^{'}_{i j}= \frac{\left\langle p_i - p_j, (p_i- p_j)^{'} \right\rangle + \left\langle p_i-p_j, (p_i - p_j)^{'}\right\rangle}{2 d_{i j}}
$$
$$
or, d^{'}_{i j}= \frac{2 \langle p_i - p_j, (p_i- p_j)^{'} \rangle}{2 d_{i j}}
$$
From scalar multiplication of inner product, we know: $$ \left\langle \alpha u, v\right\rangle =\alpha \left\langle u,v \right\rangle $$ for  vectors $u$, $v$ and scalar $\alpha$
Therefore, we finally get,
$$
d^{'}_{i j}= \left\langle \frac{ p_i - p_j}{d_{i j}}, (p_i- p_j)^{'} \right\rangle
$$
Multiplying both side with -1,
$$
-d^{'}_{i j}= \left\langle \frac{ -(p_i - p_j)}{d_{i j}}, (p_i- p_j)^{'} \right\rangle
$$
This is the hidden steps from second last step to last step of inner product computation above.





\chapter{Further discussion}
We can extend this thesis further. We can further check(study) like in Csikos's paper [7], that the differentiability condition can be replaced by continuity.\\


So now we came to the conclusion of this thesis. We proceeded this thesis by dividing this into two main parts i.e. (1) Literature review and (2) Main results. We started with the introduction of Kneser and Poulsen conjecture and talked about its importance. We described the configurationss of points under which the conjecture is constructed. We then moved to the history part. We noticed lots of Mathematicians from long time ago has been interested in this problem. We see solving this problem has different implications and helps us to think strengthen our thinking about higher dimensional space. In our thesis, we mainly followed the two papers one by Csikos[Csi98] and another by Bezdek and Conelly[BCO2]. We already talked about the importance of those paper and what has been done there. Our main problem was to follow the same methods like Csikos did and find out the formula for intersection of the balls instead of union. 

Our main results comes from the structure of the Voronoi diagram. In intersection, we take farthest Voronoi point, so if we construct in figure, the outward normal vector field from the wall of separation is in inwards towards the Voronoi cell being considered , so scaled by negative value. This will gives us negative result. There are some calculations related to t-derivatives and computing integrals. We also understood the inequalities in the lemma and understood corollaries.  We found studying the history of the development of the problem very interesting.  At last, we calculated the formula for finding the volume of intersection of balls.







\chapter*{References}\\
\begin{enumerate}
\item [1] R. Alexander, Lipschitzian mappings and total mean curvature of polyhedral surfaces, no. I, Trans. Amer.Math. Soc. 288 (1985), no. 2, 661–678\\
\item [2] M. Bern, A. Sahai, Pushing disks together—the continuous-motion case, Discr. Comput. Geom. 20 (1998),no. 4, 499–514.\\
\item [3] K. Bezdek and R. Connelly. Pushing disks apart the Kneser-Poulsen conjecture in the plane.
J. Reine Angew. Math., 553:221–236, 2002\\
\item [4] B. Bollobás, Area of the union of disks, Elem. Math. 23 (1968), no. 4, 60–61.\\
\item [5] V. Capoyleas, On the area of the intersection of disks in the plane, Comput. Geom. 6 (1996), no. 6, 393–396.\\
\item [6] V. Capoyleas, J. Pach, On the perimeter of a point set in the plane, Discrete and computational geometry (New Brunswick, NJ), DIMACS Ser. Discr. Math. Theoret. Comput. Sci., Amer. Math. Soc., Providence,RI, 6 (1991), 67–76.\\
\item [7] B. Csikós, On the volume of the union of balls, Discr. Comput. Geom. 20 (1998), no. 4, 449–461\\
\item [8] B. Csikós, On the Hadwiger-Kneser-Poulsen conjecture, Intuitive geometry, Bolyai Soc. Math. Stud. 6, János Bolyai Math. Soc., Budapest (1995), 291–299.\\
\item [9] I.Gorbovickis,"Strict Kneser–Poulsen conjecture for large radii." Geometriae Dedicata 162.1 (2013): 95-107.\\
\item [10] I.Gorbovickis, "The central set and its application to the Kneser–Poulsen conjecture." Discrete & Computational Geometry 59.4 (2018): 784-801.\\
\item [11] I.Gorbovickis, "Kneser-Poulsen conjecture for a small number of intersections." arXiv preprint arXiv:1006.0529 (2010).\\
\item [12] M. Gromov, Monotonicity of the volume of intersections of balls, Geometrical aspects of functional analysis 85–86, Springer, Lect. Notes Math. 1267 (1987), 1–4\\
\item [13] H. Hadwiger, Ungelöste Probleme No. 11, Elem. Math. 11 (1956), 60–61\\
\item [14] V. Klee, V., S. Wagon, Old and new unsolved problems in plane geometry and number theory, 11. Mathematical Association of America, Washington, DC, The Dolciani Mathematical Expositions (1991), 16–20,86–90, 274–278\\
\item [15] M. Kneser, Einige Bemerkungen über das Minkowskische Flächenmass, Arch. Math. 6 (1955), 382–390.
\item [16] E. T. Poulsen, Problem 10, Math. Scand. 2 (1954), 346.\\
\item [17] H. Edelsbrunner, The union of balls and its dual shape, Discr. Comput. Geom. 13 (1995), no. 3–4, 415–440\\
\item [18] R. Seidel, Exact upper bounds for the number of faces in d-dimensional Voronoi diagrams, DIMACS Ser.Discrete Math. Theoret. Comput. Sci., Amer. Math. Soc., Appl. geom. discr. math. 4 (1991), 517–529\\

\end{enumerate}









\end{document}




