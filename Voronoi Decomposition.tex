Voronoi(and Dirichlet)Decomposition  is a technique that divides a space into regions on the basis  of distances to a set of considered points. In chapter 4 above, we find out the method to find derivative of the function $V(t)=V(\mathbf{P}(t), \mathbf{r})$. There, we got a final formula. If we need to compute the volume of Union or intersection of balls , we first need to to divide the space using Dirichlet Voronoi decomposition and we apply Gauss-Ostrogradskii divergence theorem to those decomposed voronoi region(cells).
We will give introduction of Voronoi decomposition below.\\

\begin{itemize}


    \item Let us begin by defining the power of a point.The power of a point $\mathbf{a}$ with respect to a ball $B(\mathbf{p}, r)$ can be defined as this real number $|\mathbf{p}-\mathbf{a}|^{2}-r^{2}$. Then, when we are given  $N$ spheres by assigning the radii $\mathbf{r}$ to the centers $\mathbf{P}$. The, we can define $K_{i}(\mathbf{a})=\left|\mathbf{p}_{\mathbf{i}}-\mathbf{a}\right|^{2}-r_{i}^{2}$ as the ith  power of $\mathbf{a}$ wrt $i$ th ball. 


    
    \item  
    If $\mathbf{p}_{\mathbf{i}} \neq \mathbf{p}_{j}$,  $1 \leq i<j \leq N$, then  a half-space bounded by hyperplane is defined by this inequality $K_{i}(\mathbf{a}) \leq K_{j}(\mathbf{a})$\\ 
    The following  is the equation of the hyperplane. 
    $$
2\left\langle\left(\mathbf{p}_{\mathbf{i}}-\mathbf{p}_{j}\right), \mathbf{a}\right\rangle=\mathbf{p}_{i}^{2}-\mathbf{p}_{j}^{2}+r_{j}^{2}-r_{i}^{2} .
$$\\
Remark: Since, at hyperplane $K_{i}(\mathbf{a})=K_{j}(\mathbf{a})$,therefore this equation of the hyperplane is given by $\left|\mathbf{p}_{\mathbf{i}}-\mathbf{a}\right|^{2}-r_{i}^{2}=\left|\mathbf{p}_{\mathbf{j}}-\mathbf{a}\right|^{2}-r_{j}^{2}$


Case 1: For any pair $1 \leq i<j \leq N$, if $\mathbf{p}_{i}=\mathbf{p}_{j}$ (same center), then $K_{i}(\mathbf{a})-K_{j}(\mathbf{a})=r_{j}^{2}-r_{i}^{2}$ is constant.\\
Case 2: If balls are distinct,we determine Voronoi decomposition.


If the balls are distinct, then the balls determine a decomposition of the space into $N$ possibly unbounded or empty polyhedral domains. The $i$ th cell of the decomposition is given as the closure of the open domain















\item Then, let us suppose we are given configuration $ \mathbf{P}=(p_1, \dots ,p_N) \in  \mathbb{E}^n$ and radii is given by $r_1, \dots, r_N$. Then the following sets gives the  $i$th cell decomposition. This is called as Dirichlet Voronoi Decomposition:\\ 
$$
C_{i}=\left\{\mathbf{a} \in \mathbf{E}^{n} \mid K_{i}(\mathbf{a})\leq K_{j}(\mathbf{a}) \text { for all } 1 \leq j \leq N, j \neq i\right\} \text {. }
$$
$$
C_{i}=\left\{\mathbf{a} \in \mathbf{E}^{n} \left|\mathbf{p}_{\mathbf{i}}-\mathbf{a}\right|^{2}-r_{i}^{2}\leq \left|\mathbf{p}_{\mathbf{j}}-\mathbf{a}\right|^{2}-r_{j}^{2}  \text { for all } 1 \leq j \leq N, j \neq i\right\}
$$
We call the set $C_{i}$ as nearest point Dirichlet Voronoi region of points $\mathbf{a}$. This corresponds with Union  problem. A very good argument about how this fits the Union version of this problem is done by Edelsbrunner[17] in his paper.  




$$
C^{i}=\left\{\mathbf{a} \in \mathbf{E}^{n} \mid K_{i}(\mathbf{a}) \geq K_{j}(\mathbf{a}) \text { for all } 1 \leq j \leq N, j \neq i\right\} \text {. }
$$
$$
C^{i}=\left\{\mathbf{a} \in \mathbf{E}^{n} \left|\mathbf{p}_{\mathbf{i}}-\mathbf{a}\right|^{2}-r_{i}^{2} \geq \left|\mathbf{p}_{\mathbf{j}}-\mathbf{a}\right|^{2}-r_{j}^{2} \text { for all } 1 \leq j \leq N, j \neq i\right\}
$$
We call the set $C^{i} $ as farthest point Dirichlet Voronoi cells. This corresponds with the intersection version of our problem. More  reasearch about this region is done by Seidel[18]. This decomposition for the intersection is what we are using in our problem as we need to find the volume of intersection of the balls. 

\item When this Voronoi decomposition is induced in the ball $B(\mathbf{P}, \mathbf{r})$, we call it  (trauncated) nearest and farthest Voronoi cells for union and intersection respectively.

$$
\tilde{C}_{i}=\bar{C}_{i} \cap B(\mathbf{P}, \mathbf{r})=\bar{C}_{i} \cap B\left(\mathbf{p}_{i}, r_{i}\right),
$$
Here, $\bar{C}$ = closure of the set $C$.
$$
\tilde{C}^{i}=\bar{C}^{i} \cap B(\mathbf{P}, \mathbf{r})=\bar{C}_{i} \cap B\left(\mathbf{p}_{i}, r_{i}\right),
$$
Here, $\bar{C}$ = closure of the set $C$.




\item We mainly talk about the intersection part now because that is what we are doing later.  The term  $W^{i j}$ appear in our final formula. It is called the wall. We are considering the intersection case, so, the wall $W^{i j}$ between the cells $\tilde{C}^{i}$ and $\tilde{C}^{j}$  can be defined as the intersection of $ith$ and $jth $ vornoi regions i.e. $W^{i j}=\tilde{C}^{i} \cap \tilde{C}^{j}$. At $K_{i}(\mathbf{a})=K_{j}(\mathbf{a})$, the $W^{i j}$ is not empty. 
For easiness of notation and to show  relation of $\tilde{C}^{i}$ and $W^{i j}$ with $(\mathbf{P}, \mathbf{r})$, we normally use $\tilde{C}^{i}(\mathbf{P}, \mathbf{r})$ and $W^{i j}(\mathbf{P}, \mathbf{r})$


\item There are many easily verified properties of Voronoi diagrams. We will use mainly property(iv) given in section 3 of [BCO2] in our problem of intersection. The property says, when $W^{i j}(\mathbf{P},\mathbf{r})$ is non empty, the vector $\mathbf{p}_j - \mathbf{p}_i $ is a scalar multiple  (and negative in sign ) of the outward pointing normal to boundary of $C^i(\mathbf{r})$ at $W^{i j}(\mathbf{P}, \mathbf{r})$. If we draw  the outer unit normal vector of of region $C^{i}$ in all directions, one relating with  $C^{i}$ will be in opposite direction to the cell $C^{i}$.

\end{itemize}



